
\chapter{Shallow water '06 experiment }

The multi-institutional SW06 experiment was conducted on the New
Jersey continental shelf from July to September 2006 at a location
where internal wave activity has been observed and studied in the
past.4 In this paper, we focus on a particular internal wave event
on August 17, 2006, for which a complete set of acoustic and
environmental data were collected simultaneously. In addition, ISW
surface signatures were captured continuously by the on-board radars
of two research vessels prior to the arrival of, and during the
passing of, the ISW packet over the acoustic track. The acoustic
wave field variation is studied during this process. Other studies
of the SW06 data in relation to three dimensional modeling are being
conducted separately.

During SW06, both acoustic and environment data were collected
simultaneously. In this paper we discuss the acoustic data from a
particular acoustic source (NRL 300Hz) on the mooring denoted SW45
(see Fig. 1). The source was located 72 m below the sea surface and
10.5 m above the sea floor at $39^{o}10.957'$ N, $72^o56.575'$ W, and
transmitted 2.0489 second Linear Frequency Modulating (LFM) signals
from 270 to 330 Hz every 4 seconds. Transmissions continued for 7.5
minutes and then repeated every half hour.  A vertical and
horizontal receiver array (the "Shark VHLA") on mooring SW54 was
located at $39^o01.252'$ N, $73^o02.983'$ W, about 20.2 km south of the
NRL source.  The vertical part of the receiver array consisted of 16
hydrophones with 3.5 m spacing and was positioned in the water
column from 13.5 m to 77.75 m below the surface. The horizontal part
of the array consisted of 32 hydrophones on the seafloor with
spacing of 15 m, providing 478 m of horizontal aperture. The
sampling rate of the array was 9765.625 Hz. The acoustic track and
the locations of the source and receiver as well as the horizontal
and vertical array configurations are shown in Fig. 1. The water
depth along the acoustic track was about 80 m.
