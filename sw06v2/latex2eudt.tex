%\documentstyle [12pt] {newudthesis}
%\documentstyle [11pt] {newudthesis}
%\documentstyle [10pt] {newudthesis}
\documentclass {udthesis}
\gdef\firstchap{1}
%\nofiles
\parskip 5.5pt
\voffset -.25truein
\newcommand{\lbar}{\hbox to \hsize{\hrulefill}}
\textwidth 6.25truein
\textheight 9in
\oddsidemargin 0truein 
\newcommand{\BIBTeX}{B{\sc ib}\TeX}
\newcommand{\boxit}[1]{\vbox{\hrule\hbox{\vrule\kern3pt
	\vbox{\kern3pt#1\kern3pt}\kern3pt\vrule}\hrule}}
\def\sm{\$\$}
\def\til{\tt \char'176}
\catcode`\!=\active
	\def!#1!{{\tt \char'134 #1}}		% Use !cs! for \tt\cs
\def\uncatcodespecials{\def\do##1{\catcode`##1=12 }\dospecials}
\def\setupverbatim{\tt \def\par{\leavevmode\endgraf} \catcode`\`=\active
				   \obeylines \uncatcodespecials \obeyspaces 
                   \parskip 0pt\normalbaselines\baselineskip=12pt}
{\obeyspaces\global\let =\ }
{\catcode`\`=\active \gdef`{\relax\lq}}
\def\bverbatim{\begingroup\setupverbatim\doverbatim}
\def\doverbatim#1{\def\next##1#1{##1\endgroup}\next}

\makeatletter
\def\@verbatim{\setupverbatim}
\makeatother

\def\sskip{\vskip 3pt plus 1pt  minus 1pt}
\def\mskip{\vskip 6pt plus 2pt  minus 2pt}
\def\bskip{\vskip 12pt plus 4pt  minus 4pt}

\makeatletter
\renewenvironment{thereferences}{\begin{center}\bf REFERENCES\end{center}
 \begingroup
 \clubpenalty=10000
 \widowpenalty=10000
 \def\baselinestretch{1}\large\normalsize 
 \parindent 0pt
 \everypar{\hangindent3em}}%
{\endgroup}

\def\l@chapter#1#2{\pagebreak[3] 
 \@tempdima 1.5em 
 \let\previouslevel=\currentlevel
 \let\currentlevel=0
 \vspace*{\baselineskip}
 \begingroup
 \parindent \z@
 \parfillskip -\@pnumwidth 
 \if\firstchap0{\bf Chapter}\\
 \vspace*{\baselineskip}\gdef\firstchap{1}%
	\else\relax\fi
 \hangindent\@tempdima
 \leavevmode
 {\uppercase{\bf #1}}\nobreak\leaders\hbox{$\m@th \mkern \@dotsep mu.\mkern
 \@dotsep mu$}\hfill \nobreak \hbox to\@pnumwidth{\bf\hfil #2}\par
 \endgroup}

\def\endnotes#1{\@restonecolfalse\if@twocolumn\@restonecoltrue\onecolumn
 \fi\section*{#1}%\addcontentsline{toc}{section}{#1}
 \vspace*{\baselineskip}
 \immediate\write\loe{\noexpand\end{enumerate}}
 \immediate\closeout\loe 
 \begingroup\let\linebreak=\space\parskip=0pt
 \def\baselinestretch{1}\large\normalsize 
%\def\labelenumi{$^{\arabic{enumi}}$}
 \input\jobname.loe\relax\endgroup\initloe
 \@addtoreset{footnote}{chapter} 
 \if@restonecol\twocolumn\fi}

%\def\l@section{\relax}
%\def\l@subsection{\relax}

\makeatother

\begin{document}

\pagenumbering{roman}

\chapter*{University of Delaware\\
          Information Technologies\\ 
          User Services}

\centerline {\large\bf The \LaTeX 2e UDThesis Format}
\vspace{.5in}

\noindent {\bf Publication Date}:  {June 11, 1997}
\vspace{.5in}


\section*{Abstract}
\addcontentsline{toc}{section}{Abstract}
\noindent The \LaTeX\ UDThesis format is a document type designed to
format theses and dissertations according to the regulations specified
by the University of Delaware Office of Graduate Studies.  It provides
the correct margin width (with a bit to spare so that photocopying at
98\% doesn't cause the margins to be incorrect), as well as the proper
heading and page number placement.  \LaTeX\ UDThesis also includes
commands for automatic formatting of the title and approval page of
your document and the facility to cite bibliographic references from
your own list or from a separate database.

\noindent {\bf Note: }It is the student's responsibility to make sure
that the formatted document meets the requirements specified by the
Office of Graduate Studies' Thesis and Dissertation Manual 1995--1996, 
available in the Manuals
section at the University of Delaware Bookstore.


\thispagestyle{empty}
\newpage
\tableofcontents
\newpage

\section*{Additional Documentation}\label{adddoc}
\addcontentsline{toc}{section}{Additional Documentation}
\begin{itemize}\renewcommand{\baselinestretch}{1}\large\normalsize

\item \LaTeX: {\it A Document Preparation System} by Leslie
Lamport (Updated for \LaTeX 2e)---May be 
purchased at the University Bookstore upstairs in the
	  ``Computer'' section.

\end{itemize}

\section*{Support}
\addcontentsline{toc}{section}{Support}

\noindent Consultants are generally unfamiliar with this package and
can only help you
interpret the documentation.  Specialized consulting is available.  If
necessary, you will be referred to a specialist through the Consulting Room.
Problems with the software should be reported to the Consulting Room.

\section*{Conventions Used in This Document}
\addcontentsline{toc}{section}{Conventions Used in This Document}

\noindent Before you try to follow any of the directions in this document, you
will need to familiarize yourself with the 
conventions used in it.

The examples (set off by bars) show you how to type specific commands.
If the example uses braces \{\}, you are to type the braces.  Optional
parts of commands are set off by brackets [~].  If you use an option in
a command, you must type the brackets too.  Italicized words indicate
material that you must supply (a filename, for example).  Comments that
explain commands are set off by \% signs.  Anything that follows a \%
on the same line will not print.

\vspace*{\baselineskip}
\begin{sloppypar}
\boxit{\baselineskip14pt\copyright 1997 University of Delaware. 
Permission to copy without fee
all or part of this material is granted provided that the copies are not made
or distributed for direct commercial advantage and that the material
duplicated is attributed to Information Technologies User Services, University of
Delaware.  To copy otherwise or to republish requires specific permission
and/or a fee.
\vspace*{12pt}}
\end{sloppypar}

\newpage
\setcounter{page}{1}
\pagenumbering{arabic}

\chapter*{Typical Manuscript File Organization}
\addcontentsline{toc}{chapter}{Typical Manuscript File Organization}

\noindent To produce a large document, use three types of files:

\begin{itemize}\itemsep0pt\parskip0pt\itemindent6em
\item main manuscript file
\item style file
\item front material, chapter, and appendix files
\end{itemize}

It is generally most convenient to put each chapter or appendix
into a separate file. These files must each have an 
extension of {\bf .tex}.
To include the text of the
files in the correct place in the completed document, 
type !include! or !input! (see {\bf Running One Chapter at a Time} for
an explanation of the differences) followed by the filename
enclosed in \{\} in the main manuscript file.  
When you include the files in the main manuscript
file, use only the
filename without the {\bf .tex} extension. 
For example, to include the file for chapter 1 
(which you have called {\bf chapter1.tex}),
you would type

\mskip
!include!\{chapter1\}
\mskip

\noindent The {\bf main mainuscript file} should be organized in this manner:

\vspace*{\baselineskip}

{\baselineskip 12pt
\moveright\parindent\hbox{\vbox{
\hbox{!documentclass! \{udthesis\}}
\hbox{!begin! \{document\}}
\hbox{!include!\{{\it filename for title and approval page, abstract, etc.}\}}
\hbox{!include!\{{\it filename for first chapter}\}}
\hbox{!include!\{{\it filename for second chapter}\}}
\hbox{...}
\hbox{!include!\{{\it filename for first appendix}\}}
\hbox{!include!\{{\it filename for second appendix}\}}
\hbox{...}
\hbox{!include!\{{\it filename for references}\}}
\hbox{!end! \{document\}}}}}

The first line of your file must be the 
\verb+documentclass {udthesis}+ command.  
If you do not include a font size after
!documentclass!, or you try to specify 11-point font, your document
will be printed in the 12-point font regardless.

\noindent A typical input file for a chapter or appendix might look like this:

\mskip
\begin{verbatim}
\chapter {Introduction}
\section {Section One}
\end{verbatim}
\mskip

The included file for the title and approval page and other front 
material might look like this (comments are set off by \% signs):

\lbar
\begin{verbatim}
(Title Page commands)     % see Title and Approval Page section
\maketitlepage            % make title page 
\begin{approvalpage}      % begin approval page
(Approval Page commands)  % see Title and Approval Page section
\end{approvalpage}        % end approval page
\begin{signedpage         % begin signature page (Dissertation only)
(Signature Page commands) % see Title and Approval Page section
\end{signedpage}          % end signature page

\begin{front}             % begin front material
\prefacesection{title for acknowledgments} 
   (text)
\maketocloflot            % create table of contents, 
                          % list of figures, and list of tables 
\prefacesectiontoc{title for abstract}
   (text)
\prefacesectiontoc{title for summary}
   (text)
\end{front}               % end front material
\end{verbatim}
\lbar

{\bf Note: } If you do not need to generate a list of figures, put
\verb+\figurespagefalse+ before \verb+\maketocloflot+.  If you do not
need to generate a list of tables, put \verb+\tablespagefalse+ before
\verb+\maketocloflot+.

\newpage
\section*{Running One Chapter at a Time}
\addcontentsline{toc}{section}{Running One Chapter at a Time}

The best way to run a chapter by itself is to use the \verb+\include+ macro
rather than \verb+\input+. You can tell \LaTeX\ which file to include
by using the \verb+\includeonly+ macro.  The advantage with this method
is the ability to retain the correct chapter and page numbers for the
files you choose to format.  Here is a typical main manuscript 
file, using this method.

\lbar
\begin{verbatim}
\documentclass {udthesis}
\includeonly{tap}
\begin{document}
\include{tap}
\include{chap1}
\include{chap2}
\include{chap3}
\include{chap4}
\include{appA}
\include{appB}
\end{document}
\end{verbatim}
\lbar

The above example would format the file {\bf tap.tex}, which is the
Title and Approval pages.  If you use this method, you must format the
files in order at least once to set up the correct chapter and page
numbers.  Please refer to the User Services' document
\LaTeX\ Advanced Topics: ``Separating Files'' on page 22 or Leslie 
Lamport's \LaTeX\ book: ``Splitting Your Input'' on page 75 for
additional help.


\chapter*{Automatic Numbering}
\addcontentsline{toc}{chapter}{Automatic Numbering}
\noindent \LaTeX\ automatically generates numbers for the sections of your 
document.
Following are the sectioning commands you can use in UDThesis and the results
they
will produce in your text.
\section*{Sectioning Commands}
\addcontentsline{toc}{section}{Sectioning Commands}

\vspace*{\baselineskip}
\begin{center}
\renewcommand{\arraystretch}{.75}
\begin{tabular}{|l||l|}\hline
!chapter!&automatically numbers chapters; you supply chapter title\\
&e.g., \verb+\chapter{Introduction}+\\
!chapter*!&same as \verb+\chapter+, but unnumbered \\
& and no Table of Contents entry\\
!appendix!&same as \verb+\chapter+, but for appendices\\
&e.g., \verb+\appendix{Computer Programs}+\\
!oneappendix!&same as \verb+\appendix+, but used for only one appendix\\
!prefacesection!&same as \verb+\chapter*+, but used only in the front material
\\ &and no Table of Contents entry\\
&e.g., \verb+\prefacesection{Acknowledgments}+\\
!prefacesectiontoc!&same as \verb+\prefacesection+, but there is\\
&a Table of Contents entry\\
&e.g., \verb+\prefacesectiontoc{Abstract}+\\
!section!&numbered within Chapter or Appendix\\
&e.g., \verb+\section{Melt Down}+\\
!section*!&same as \verb+\section+, but unnumbered\\
!subsection!&numbered within Section\\
&e.g., \verb+\subsection{The First Minute After}+\\
!subsection*!&same as \verb+\subsection+, but unnumbered\\
!subsubsection!&numbered within SubSection\\
&e.g., \verb+\subsubsection{The First Day After}+\\
!subsubsection*!&same as \verb+\subsubsection+, but unnumbered\\\hline
\end{tabular}
\end{center}


\subsection*{Putting an Unnumbered Entry in the Table of Contents} 
\addcontentsline{toc}{subsection}{Adding an Unnumbered Table of Contents Entry} 

If you use the \verb+*+ form of the sectioning commands, there will be no entry made in the
Table of Contents.  If you would like an unnumbered entry in the Table of
Contents, use

\mskip
\verb+\addcontentsline{toc}+\{{\it section type}\}\{{\it title}\}
\mskip

\noindent where {\it section type} is the name of one of the sectioning commands
without
the \verb+\+ and {\it title} is the text to be entered into the Table of
Contents.  It is most convenient to include the \verb+\addcontentsline{toc}+
command immediately after each sectioning command in your thesis.  Remember that
you only need to do this if you use the \verb+*+ form of the sectioning command.
For example,

\begin{verbatim}
\section*{All Figures Representing X,Y,Z Data}
\addcontentsline{toc}{section}{All Figures Representing X,Y,Z Data}
\end{verbatim}

\subsection*{Breaking a Long Title}
\addcontentsline{toc}{subsection}{Breaking a Long Title}
If a chapter title is long and you need to break it, use the
\verb+\protect\linebreak+
command after the word where you want to break the line.  For example

\begin{verbatim}
\chapter{This is a very long title\protect\linebreak
         that will be centered properly}
\end{verbatim}

\section*{Numbered and Unnumbered Equations}
\addcontentsline{toc}{section}{Numbered and Unnumbered Equations}

\begin{enumerate}
\item  {\bf Numbered Equations}

The {\bf equation} and {\bf eqnarray} math environments generate
numbers that can be assigned to a key with
a {\bf !label!} command. 
The environments 

\begin{center}
\begin{tabular}{l}
\verb+\begin{equation}+ {\it formula} \verb+\end{equation}+\\
\verb+\begin{eqnarray}+ {\it formula} \verb+\end{eqnarray}+
\end{tabular}
\end{center}
 
\noindent will put the equation number flush right 
with the margin, unless the {\bf leqno} document style option is used. Each
equation number is produced within a chapter or appendix.  See 
page~\pageref{equ1} for an example.

\item {\bf Unnumbered equations}

There are two types of unnumbered equations:

\begin{enumerate}
\item {\bf in-text}  A formula that appears in the middle of a sentence is
called an {\bf in-text}
formula.  There are three ways to produce an in-text formula:

\vspace*{.25\baselineskip}
\begin{tabular}{@{\hspace{.5in}}l}
\$ {\it formula} \$ \\
\verb+\+( {\it formula} \verb+\+)\\
\verb+\begin{math}+ {\it formula} \verb+\end{math}+\\
\end{tabular}
\vspace*{.25\baselineskip}
\item {\bf displayed} A displayed formula appears outside of the running text.
Displayed equations
are centered and unnumbered.  These can be produced in three ways:

\vspace*{.25\baselineskip}
\begin{tabular}{@{\hspace{.5in}}l}
\$\$ {\it formula} \$\$\\
\verb+\+[ {\it formula} \verb+\+]\\
\verb+\begin{displaymath}+ {\it formula} \verb+\end{displaymath}+\\
\end{tabular}
\vspace*{.25\baselineskip}
\end{enumerate}
\end{enumerate}

\section*{Footnote Numbers}
\addcontentsline{toc}{section}{Footnote Numbers}

\noindent Footnotes will be numbered automatically by using the 
\verb+\fnote+ command. To create a footnote, use 

\verb+\footnote+\{{\it text of footnote}\}

\lbar

\noindent {\bf What you type:}

\begin{verbatim}
This is a footnote.\footnote{What is this all about?} Now how
about a second one.\footnote{Isn't this easy?}
\end{verbatim}

\noindent {\bf Result:}

This is a footnote.\footnote{What is this all about?} Now how
about a second one.\footnote{Isn't this easy?}

\lbar

\newpage
\section*{Endnote Numbers}
\addcontentsline{toc}{section}{Endnote Numbers}
\setcounter{footnote}{0}
\noindent Endnotes will be numbered automatically by using the
\verb+\enote+ command. To create an endnote, use

\verb+\enote+\{{\it text of endnote}\}

\lbar

\noindent {\bf What you type:}

\begin{verbatim}
This is an endnote.\enote{You should not mix footnotes and endnotes.} 
Now how about a second one.\enote{Use one or the other.}
\end{verbatim}

\noindent {\bf Result:}

This is an endnote.\enote{You should not mix footnotes and endnotes.} 
Now how about a second one.\enote{Use one or the other.}

\lbar

To produce the list of endnotes, use \verb+\endnotes+\{{\it Title}\}
command at the end of each chapter or 
at the end of your entire document in {\bf main.tex}.  For example,

\lbar

\noindent {\bf What you type:}

\begin{verbatim}
\endnotes{Notes}
\end{verbatim}

\noindent {\bf Result:}

\endnotes{Notes}

\lbar

\newpage
\section*{Figures and Tables}
\addcontentsline{toc}{section}{Figures and Tables}

\noindent A figure or table number will be produced whenever 
a \verb+\caption+[{\it entry}]\{{\it heading}\} command
appears within a {\bf figure} or {\bf table} environment.
You can use the {\it entry} option to specify an
alternative heading for the List of Figures or List of Tables.  If you do
not use this option, the {\it heading}---the actual
text of the caption---will be used.  These environments are of the form

\verb+\begin{figure}+[{\it loc}] {\it body} \verb+\end{figure}+

\noindent or

\verb+\begin{table}+[{\it loc}] {\it body} \verb+\end{table}+

\noindent where {\it body} is the actual figure or table.  Use the 
{\it loc}
option to specify the location of the figure or table.  Possibilities
for placement are {\bf h} (here), {\bf t} (top), {\bf b} (bottom), and 
{\bf p} (page).  You should always include the {\bf p} option with any 
other options. If you do not specify a location, the default
is {\bf tbp}.   
The figure or table will be placed on the first page on which space 
is available for the entire body of your figure or table.  The number 
is generated within the chapter or appendix when ever a \verb+\caption+
is used.  Therefore you may use \verb+\caption+ more than once within
the figure or table environment.  This is especially nice if you want 2
figures or tables to be kept together on one page.
See page~\pageref{fig1} for examples of multiple captions in one figure
environment and single caption in a table environment.

You may find it difficult to convince \LaTeX\ to put your figure or
table where you want it.  It is best to place the figure or table
environment commands close to where you think the top of a new page will
start. 

{\bf Note:}  This environment is meant to create a floating area within
your document.  It does NOT actually create a table or figure, you need
to use the {\bf tabular} environment for tables (Leslie Lamport's book,
pages 63 and 182) and the {\bf picture} environment for figures (Leslie
Lamport's book, pages 101 and 196).  These other environments are part
of \LaTeX, however you can use other graphics packages to import
graphics.  These graphics must be compatible with your printer.


\subsection*{Full Page Figures and Tables}
\addcontentsline{toc}{subsection}{Full Page Figures and Tables}

To create a full page figure or table you need to use \verb+\vbox+ to
make an imaginary box the height of the entire page.  
For normal full page figures and tables, where a figure or table will not
be pasted in later, use the following method.  The figure or table will
be centered vertically on the page.

\noindent {\bf Full Page Figure:}

\lbar
\begin{verbatim}
\begin{figure}
\vbox to \textheight{%
\vfill
[the body of the figure goes here]
\caption{This is My Full Page Figure}
\vfill}
\end{figure}
\end{verbatim}
\lbar

\noindent {\bf Full Page Table:}

\lbar
\begin{verbatim}
\begin{table}
\vbox to \textheight{%
\vfill
\caption{This is My Full Page Table}
[the body of the table goes here]
\vfill}
\end{table}
\end{verbatim}
\lbar

\newpage

The examples below
show how to create a full page figure or table that will be pasted in later.

\subsection*{Paste-in Full Page Figure}
\addcontentsline{toc}{subsection}{Paste-in Full Page Figures and
Tables}

The top of the page is blank with whe caption is flush at the bottom of
the page.

\lbar
\begin{verbatim}
\begin{figure}
\vbox to \textheight{%
\vfill
\caption{This is Paste-in My Full Page Figure}}
\end{figure}
\end{verbatim}
\lbar

\subsection*{Paste-in Full Page Table}

The caption is flush at the top of the page and the rest of the page is
blank.

\lbar
\begin{verbatim}
\begin{table}
\vbox to \textheight{%
\caption{This is Paste-in My Full Page Table}
\vfill}
\end{table}
\end{verbatim}
\lbar


\newpage

\chapter*{Creating Cross-References in UDThesis Format}
\addcontentsline{toc}{chapter}{Creating Cross-References in UDThesis Format}

\noindent All referencing is done by using the 
\verb+\label+\{{\it key}\}
command where {\it key} is
the word you use
to refer to the number of the chapter,
appendix, section, subsection, subsubsection, equation, figure, or table.
Once you have assigned this key, you can refer to the number (of the appropriate
section) by using
\verb+\ref+\{{\it key}\}
and to the appropriate page by using
\verb+\pageref+\{{\it key}\}
(to refer to the page number).
These commands do not insert text before the numbers.  This means that you must
insert the word ``chapter,'' ``appendix,'' ``section,'' ``subsection,''
``subsubsection,'' ``equation,'' ``figure,'' or ``table'' before the 
\verb+\ref+ and \verb+\pageref+ commands.
You should use a ``{\bf\til}'' between the proper word and
{\bf !ref!} and {\bf !pageref!} to prevent a line break from occurring between
the word and the number that follows it.
There can be no spaces before or after the ``{\bf\til}''
(see examples below).

{\bf Note:} If you want a reference to be within another macro, like
\verb+\chapter+, \verb+\section+, \verb+\caption+, etc., then you must use 
\verb+\protect+ before that reference command.   For example

\lbar

\begin{verbatim}
\begin{figure}
\vspace*{4.5in}
\caption{Referencing another figure, see figure\protect\ref{fig1}}
\label{fig3}
\end{figure}
\end{verbatim}

\lbar

\newpage
\section*{Referring to a Chapter, Appendix or Section}
\addcontentsline{toc}{section}{Referring to a Chapter, Appendix, or Section}
\noindent The next example illustrates references to a section and a subsection.
References to all other sectioning commands would work in the same way.

\lbar
\noindent {\bf What you type}:

\begin{verbatim}

\section{This is the first section}
\label{sec1}
In section one, we will present our hypothesis.

\subsection{This is the first subsection}
\label{ssec1}
In section~\ref{sec1} on page~\pageref{sec1}, you will 
find the statement of our hypothesis. It is explained more fully in 
section~\ref{ssec1} on page~\pageref{ssec1}.
\end{verbatim}

\noindent {\bf Result}:

\setcounter{chapter} {1}
\section{This is the first section}
\label{sec1}
In section one, we will present our hypothesis.
\subsection{This is the first subsection}
\label{ssec1}
In section~\ref{sec1} on page~\pageref{sec1}, you will 
find the statement of our hypothesis.  It is explained more fully in 
section~\ref{ssec1} on 
page~\pageref{ssec1}.

\lbar

\newpage
\section*{Referring to Tables and Figures}
\addcontentsline{toc}{section}{Referring to Tables and Figures}

\noindent According to the Graduate School guidelines,
figure numbers and captions must be {\bf below} the figure,
while table numbers and 
captions must be {\bf above} the table.
The following example shows how to create and refer to a figure. 

\lbar
\noindent {\bf What you type}:

\begin{verbatim}
\begin{figure}[h]
[The body of the first figure goes here.]
\caption{Circuit Diagram I - 1935}
\label{fig1}
\vspace*{1.5in}
[The body of the second figure goes here.]
\caption{Circuit Diagram I - 1936}
\label{fig2}
\end{figure}

Figure~\ref{fig1} on page~\pageref{fig1} shows the circuit diagram used
for the preliminary SDI research in 1935 and figure~\ref{fig2} on
page~\pageref{fig2} in 1936.
\end{verbatim}

\noindent {\bf Result}:

\begin{figure}[h]
\centerline{[The body of the figure goes here.]}
\caption{Circuit Diagram - 1936}
\label{fig1}
\vspace*{1.5in}
\centerline{[The body of the second figure goes here.]}
\caption{Circuit Diagram I - 1936}
\label{fig2}
\end{figure}

Figure~\ref{fig1} on page~\pageref{fig1} shows the circuit diagram used
for the preliminary SDI research in 1935 and figure~\ref{fig2} on
page~\pageref{fig2} in 1936.

\lbar

\newpage
The next example shows how to create and refer to a table. 

\lbar
\noindent {\bf What you type}:

\begin{verbatim}

\begin{table}[h]
\caption{Table Example - 1987 Budget Report}
[The body of the table goes here.]
\label{tab1}
\end{table}

Table~\ref{tab1} on page~\pageref{tab1} shows an example of where to place
a table caption.
\end{verbatim}

\noindent {\bf Result}:

\begin{table}[h]
\caption{Table Example - 1987 Budget Report}
\center [The body of the table goes here.]
\label{tab1}
\end{table}

Table~\ref{tab1} on page~\pageref{tab1} shows an example of where to place a
table caption.

\lbar

\section*{Referring to Equations}
\addcontentsline{toc}{section}{Referring to Equations}

\noindent To refer to an equation number, you must use one of the math
environments that
number equations.  They are

\verb+\begin{equation}+ {\it formula} \verb+\end{equation}+
\mskip
\noindent or
\mskip
\verb+\begin{eqnarray}+ {\it formula} \verb+\end{eqnarray}+

\noindent where {\it formula} is the math equation that you supply.
To define a label
for an equation number, you must put the \verb+\label+\{{\it key}\}
command on the
line of the formula to which you want to refer.
 
\newpage
The following example shows how to generate equation numbers for a single and
a multilined equation.

\lbar
\noindent {\bf What you type}:

\begin{verbatim}

\begin{equation}
x = {x \over 2} \label{equ1}
\end{equation}

\begin{eqnarray}
x & = & 17y \label{equ2}\\
y & > & a+b+c+d+ \nonumber\\
  &   & e+f+g
\end{eqnarray}

Equation~\ref{equ1} on page~\pageref{equ1} is an example of a simple
displayed equation and equation~\ref{equ2} on page~\pageref{equ2} is
an example of a multilined equation.
\end{verbatim}

\noindent {\bf Result}:

\begin{equation}
x = {x \over 2} \label{equ1}
\end{equation}

\begin{eqnarray}
x & = & 17y \label{equ2}\\
y & > & a+b+c+d+ \nonumber\\
  &   & e+f+g
\end{eqnarray}

Equation~\ref{equ1} on 
page~\pageref{equ1} is an example of a simple displayed equation and
equation~\ref{equ2} on page~\pageref{equ2} is an example of a multilined
equation.

\lbar

\chapter*{List of References}
\addcontentsline{toc}{chapter}{List of References}

\noindent In \LaTeX, you can create a bibliography or list
of references in two ways.  You
can either produce a list of sources yourself or use a separate program called
\BIBTeX\ to generate a list from information you have stored in a bibliographic
database.  Both methods are explained in detail in Leslie Lamport's \LaTeX\
book, section 4.3. 

{\bf Note:} If you want to cite a reference within another macro, like
\verb+\chapter+, \verb+\section+, \verb+\caption+, etc., then you must use 
\verb+\protect+ before that \verb+cite+ command.   For example

\lbar

\begin{verbatim}
\begin{figure}
\vspace*{4.5in}
\caption{Very important figure\protect\cite{Abr86}}
\label{fig4}
\end{figure}
\end{verbatim}

\lbar

\section*{Using \BIBTeX}
\addcontentsline{toc}{section}{Using \BIBTeX}

\noindent \BIBTeX\ is a separate program from \LaTeX.  This program creates a
{\bf .bbl} file that will be read by \LaTeX\ to create references to sources
that you
cite in your thesis with the \verb+\cite+ command.  The {\bf .bbl} file that
\BIBTeX\ creates is taken from a bibliographic database file that contains
information about your sources.  This database file must have the extension
{\bf .bib}
and must follow a specific format.  For the format, consult the \LaTeX\ book or
the \BIBTeX\ document in the Smith Consulting Center (002 Smith Hall).  Once you
have created the {\bf .bib} file, you tell \LaTeX\ to use it by including the
following command in your {\bf main.tex} file:

\bskip
\verb+\bibliography+\{{\it file}({\it s})\}
\bskip

\noindent where {\it file}({\it s}) is the name of your {\bf .bib} file or 
files.
See the example in the Appendix on page~\pageref{main2} for including the
\verb+\bibliography+ command in your {\bf main.tex} file.
If you have more than one bibliographic database file, separate each
file name with a comma.  For example, if you have two databases---{\bf acs.bib}
and {\bf ait.bib}, you would use the form \verb+\bibliography+\{acs,ait\}.

After you have included your database files in the {\bf main.tex} file, you
must select a bibliography style file with the command

\bskip
\verb+\bibliographystyle+\{{\it type}\}
\bskip

\noindent where {\it type} can be one of four styles:

\begin{itemize}
\item {\bf plain}--sorted alphabetically with numbered labels
\item {\bf unsrt}--listed in order of citing in the text; numbered labels
\item {\bf alpha}--sorted alphabetically with labels formed from author's name
and the year of publication
\item {\bf abbrv}--same as {\bf plain} except that first names, and names of
months and journals are abbreviated
\end{itemize}

To refer to the bibliography item in your text, use the \verb+\cite+ command.
If you would like to have a bibliography entry included in your list of
references but not cited in your text, you can refer to it using the
\verb+\nocite+ command.  You can place this command anywhere after the
\verb+\begin{document}+ command.

Once you have created your database file and included the appropriate
references to it in your {\bf main.tex} file and the citations in your
individual
chapter files,
you run your {\bf main.tex} file through \LaTeX, then
through \BIBTeX, and then
through \LaTeX\ again.  This process will create the {\bf .bbl} file and allow
it to be used by \LaTeX. 

\newpage
In general if you make a bibliographic database and want to use
it with \LaTeX, do the following

\begin{itemize}
\item {\bf latex} {\it filename}
\item {\bf bibtex} {\it filename}
\item {\bf latex} {\it filename}
\end{itemize}

\noindent where {\it filename} is the name of your {\bf main.tex} file.

{\bf Note:}  If you add or remove any citations in your database file,
you will have to 
rerun your {\bf main.tex} file through \LaTeX, then through \BIBTeX, and then
through \LaTeX\ again.


The following example shows the format for book entries in a {\bf .bib} file.

\lbar
\begin{verbatim}

@book(lov78,
author = "Donald W. Loveland",
title = "Automated Theorem Proving: A Logical Basis",
publisher = "North-Holland Publishing Company",
year = 1978,
address = "New York")
\end{verbatim}

\lbar

\section*{Creating a List of Sources Yourself}
\addcontentsline{toc}{section}{Creating a List of Sources Yourself}

Following is some necessary information for creating a list
of sources yourself using the {\bf thebibliography} environment.
You use the {\bf thebibliography} environment whether you want your
list of references
to be called ``REFERENCES'' or ``BIBLIOGRAPHY.''  Begin the environment with

\bskip
\verb+\begin{thebibliography}+ \{{\it label width}\}
\bskip

\noindent and end it with

\bskip
\verb+\end{thebibliography}+
\bskip

\noindent The {\bf thebibliography} environment is like the \LaTeX\ 
list environments
except that

\begin{itemize}
\item Each item begins with a !bibitem! command.  This command is followed by a
key by which the item can be cited in the text with a !cite! command.

\item The {\bf thebibliography} environment includes a {\it label width},
that is, a
number or series of letters that is the same width as or slightly wider than
the widest item label in your list of sources. (For example, if your list has
500 entries, you might choose ``999'' as your label width because ``999'' is
as wide as all other three-digit numbers.)
\end{itemize}

You can choose what to call your references section by redefining
\verb+\bibname+ with the name of the title
 before {\bf thebibliography} environment command.

The following example shows how you would construct a {\bf thebibliography}
environment, how you would cite a source in your text and change the
title of your references to ``References'' rather than ``Bibliography''.

\lbar
\begin{verbatim}

\renewcommand{\bibname}{References}
\begin{thebibliography}{999}
\bibitem{Catt1} Cattell, R.B. 1966. ``The Screen Test for the Number of
         Factors.'' {\it Multivariate Behavior Research}, 1: 245-276.
\end{thebibliography}
\end{verbatim}

\begin{verbatim}
See \cite{Catt1} for more information on the test.
\end{verbatim}

\lbar

\newpage
\chapter*{Additional Commands}
\addcontentsline{toc}{chapter}{Additional Commands}

\section*{Hanging Indent References}
\addcontentsline{toc}{section}{Hanging Indent References}

Hang indent references can be created with the {\bf thereferences}  
environment. To use this environment:

\lbar
\noindent {\bf What you type}:

\begin{verbatim}
\begin{thereferences}
Leslie Lamport, \LaTeX: {\it A Document Preparation System} (User's 
Guide and Reference Manual), 1986, Addison-Wesley, Reading, Massachusetts.

Donald Knuth, {\it The \TeX book}, 1986, Addison-Wesley, Reading,
Massachusetts.
\end{thereferences}
\end{verbatim}

\noindent {\bf Result}:

\begin{thereferences}
Leslie Lamport, \LaTeX: {\it A Document Preparation System} (User's 
Guide and Reference Manual), 1986, Addison-Wesley, Reading, Massachusetts.

Donald Knuth, {\it The \TeX book}, 1986, Addison-Wesley, Reading,
Massachusetts.
\end{thereferences}

\lbar

\noindent This will produce the title {\bf REFERENCES}; however, you can specify a
title using the option format 
\begin{center}
\begin{tabular}{l}
\verb+\renewcommand{\bibname}{Bibliography}+\\
\verb+\begin{thereferences} .. \end{thereferences}+
\end{tabular}
\end{center}

\newpage
\section*{Continuing Figures and Tables}
\addcontentsline{toc}{section}{Continuing Figures and Tables}

To continue a figure or table on another page without adding another figure
or table number, you can use the \verb+\captioncont+\{\} command as follows:

\lbar
\noindent {\bf What you type}:

\begin{verbatim}
\begin{figure}
     [The first part of the figure goes here.]
\caption{This is my picture}
\end{figure}

\begin{figure}
     [The second part of the figure goes here.]
\captioncont{continued}
\end{figure}
\end{verbatim}
\noindent {\bf Result}:

\begin{figure}[h]
\center [The first part of the figure goes here.]
\center {\bf Figure 1.1:} This is my picture
\end{figure}

\begin{figure}[h]
\center [The second part of the figure goes here.]
\center {\bf Figure 1.1:} continued
\end{figure}

\lbar
\noindent Note that the figure number is the same for the second 
figure when using \verb+\captioncont+.  If you were using the table
environment instead of the figure environment (as in the above
example), then the table number would be the same if you used 
\verb+\captioncont+.

\newpage
\section*{Glossary or List of Symbols}
\addcontentsline{toc}{section}{Glossary or List of Symbols}

To create a glossary or list of symbols a {\bf desc} environment has
been created. You can use this environment as follows:

\lbar
\noindent {\bf What you type}:

\begin{verbatim}
\prefacesectiontoc{Glossary}
\begin{desc}
\item[cat] A carnivorous mammal domesticated since early times as a catcher of 
rats and mice and as a pet.
\item[mouse]  Any of numerous small rodents of the families Muridae and 
Cricetidae, such as the common house mouse, characteristically having a long,
naked or almost hairless tail.
\item[mousetrap] A trap for catching mice.
\end{desc}
\end{verbatim}

\noindent {\bf Result}:

\begin{center}
{\large\bf GLOSSARY}
\end{center}
\begin{desc}
\item[cat] A carnivorous mammal domesticated since early times as a catcher of 
rats and mice and as a pet.
\item[mouse]  Any of numerous small rodents of the families Muridae and 
Cricetidae, such as the common house mouse, characteristically having a long,
naked or almost hairless tail.
\item[mousetrap] A trap for catching mice.
\end{desc}

\lbar
\noindent This will align all definitions after the words or symbols. A maximum
label width of one inch is provided for each item.

\chapter*{Order of the UDThesis Sections}
\addcontentsline{toc}{chapter}{Order of the UDThesis Sections}

\noindent The order in which
you generate the parts of your documents is critical.  To have your page and
section numbers
generated correctly, you need to follow
this ordering:

\vspace{.25in}

\hbox to \textwidth{\hfil\vbox{
\hbox{Title and Approval Page}
\hbox{Acknowledgments}
\hbox{Preface}
\hbox{Contents--the table of contents}
\hbox{FigureContents--the list of figures}
\hbox{TableContents--the list of tables}
\hbox{Abstract}
\hbox{Summary}
\hbox{(body of thesis, dissertation, or executive posistion paper)}
}\hfil}

{\bf Note}--The page numbers in these parts (except for the body of the
thesis) are in lower-case
Roman numerals; the page numbers throughout the remainder of the
thesis are in Arabic numerals.  This change in style is automatic and occurs
when UDThesis produces each part.

\section*{Title and Approval Page}
\addcontentsline{toc}{section}{Title and Approval Page}

\noindent UDThesis uses a series of commands to make the formatting of title
and approval
page easy.
Generally, there is some standard text
associated with each command which appears only if you include the command.
Note that all the commands are in
lower case; this is how they must appear in your document or an error
will occur.  The commands are

\newpage
\lbar
\bskip
\noindent {\bf Title Page}:
\begin{verbatim}

\title{Title of paper}
         % To indicate a new line in the title, use \linebreak
         % at the end of each line but the last. The title is converted to 
         % upper case automatically.    
\author{name of author}
\type{type of document} % e.g., thesis, dissertation or 
                        %       executive position paper
\degree{name of degree} % e.g., Master of ..., Doctor of Philosophy or
                        %       Doctor of Education 
\educationtrue or \educationfalse
         %To indicate whether or not the document is an executive 
         %position paper.
         %Default is \educationfalse.
\majorfieldtrue or \majorfieldfalse
         % To indicate whether or not a major is to be included. 
         % Default is \majorfieldfalse.
\majorfield{name of majorfield} % e.g., Physics
         % Include only if you used \majorfieldtrue.
\degreedate{date degree is to be granted} % e.g., Fall 1993
                                          %       Spring 1993
                                          %       Summer 1993
\end{verbatim}

\noindent {\bf Approval Page}:

\begin{verbatim}
\prof{name, highest degree of your thesis adviser}
         %Do not use for dissertation or executive position paper.
\prof{name, highest degree of your second thesis adviser}
         %Do not use for dissertation or executive position paper.
\chair{name, highest degree of Chair}{position (title) of Chair}         (1)
\auxchair{name, highest degree of AuxChair}{position (title) of AuxChair}(2) 
\dean{name, highest degree of Dean}{position (title) of Dean}            (3) 
\end{verbatim}

\noindent {\bf Statement and Signature Page}:(dissertation or executive
position paper only)

\begin{verbatim}
\profmember{name, highest degree of your thesis adviser}
\member{member of dissertation committee}
         %Use for each member of your committee.
\end{verbatim}

\lbar

{\bf Note for Thesis:} You must have at least three signatures on the 
Approval Page: the first must be your thesis adviser's (\verb+\prof+);
the second must be one of the following: (1) \verb+\chair+, 
(2) \verb+\auxchair+, or 
(3) \verb+\dean+; \LaTeX\ automatically places the third---Carol E. Hoffecker,
Ph.D., Associate Provost for Graduate
Studies---as the last signature on the
Title and Approval page.

{\bf Note for Dissertation and Executive Position Paper:} You do not use
\verb+\prof+ on the Approval Page, this will be included on the Signature Page.
The Signature Page requires at least four signatures: the first must be your
adviser's (\verb+\profmember+), and the rest should be the other members
(\verb+\member+) of your committee. 
This page can have at most six signatures.

\vspace*{-\baselineskip}
\subsection*{Examples for Title and Approval Page}
\addcontentsline{toc}{subsection}{Examples for Title and Approval Page}

\vspace*{-.5\baselineskip}
\noindent The following example illustrates the use of \LaTeX\
to generate the
Title and Approval Page and other information which will appear at the 
beginning of your document.  You must insert this material
at the {\bf beginning} of your document to produce the Title and Approval Page.
See the examples in the {\bf Appendix} starting on page~\pageref{exfiles} 
for more help.  

\vspace*{-.5\baselineskip}
\lbar 

\addcontentsline{toc}{subsubsection}{Thesis Example}
\noindent {\bf Thesis Title and Approval Page:} 
\begin{verbatim} 

\title{Irradiation and Annealing Effects\\
	   in Amorphous Alloys}
\author{John H. Poe}
\type{thesis}
\degree{Master of Science}
\majorfieldtrue\majorfield{Physics} % optional
\degreedate{Spring 1993}

\maketitlepage 

\begin{approvalpage}
\prof{Jon B. Nill, Ph.D.}     
\prof{K. Casey, Ph.D.} % use only if 2 advisers',
                       % otherwise, omit.
\chair{Willie B. Null, Ph.D.}{Chairman of the Department of Physics} 
\end{approvalpage} 

\begin{front} 

\prefacesection{Acknowledgments}
   \input{acknowl}

\prefacesection{Preface}
   \input{dedicat}

\maketocloflot

\prefacesectiontoc{Abstract}
   \input{abstract}

\end{front}
\end{verbatim}

\lbar
\newpage

\lbar
\addcontentsline{toc}{subsubsection}{Dissertation or Executive Position Paper
Example}
\noindent {\bf Dissertation or Executive Position Paper Title and Approval Page:} 
\begin{verbatim}

\title{Irradiation and Annealing Effects\\
	   in Amorphous Alloys}
\author{Dale S. Hoover}
\type{dissertation}
\degree{Doctor of Philosophy}
\majorfieldtrue\majorfield{Physics} % optional
\degreedate{Spring 1993}

\maketitlepage 

\begin{approvalpage} 
\chair{Willie B. Null, Ph.D.}{Chairman of the Department of Physics} 
\end{approvalpage} 

\begin{signedpage}
\profmember{John L. Smith, Ph.D.}
\member{Mary T. Less, Ph.D.}
\member{Harry S. Shipley, Ph.D.}
\member{Chang Lee, Ph.D.}
\end{signedpage}

\begin{front} 
\prefacesection{Acknowledgments}
   \input{acknowl}

\prefacesection{Preface}
   \input{dedicat}

\maketocloflot 

\prefacesectiontoc{Abstract}
   \input{abstract}

\end{front}
\end{verbatim}

\lbar


\newpage

\section*{Abstract with Title and Approval Page}
\addcontentsline{toc}{section}{Abstract with Title and Approval Page}
\noindent UDThesis provides the necessary macros to prepare an abstract
with its own title and approval page.  You should make a {\it separate
file\/} containing the necessary information and run it through
\LaTeX\ apart from the rest of the document in order to give correct
page numbers. Below is a list of the possible macros to be used
in the separate file to generate an abstract with its own title and
approval page.


\begin{verbatim}
\documentclass {udthesis}
\begin{document}
\title{document title}
\author{author's name}
\type{dissertation}
\degree{Doctor of Philosophy}
\majorfieldtrue\majorfield{major field of study} %optional
\degreedate{Spring 1993}

\begin{abtandapage}
\abprof{adviser's name}
\abprof{second adviser's name}    % use only if 2 advisers',
                                  % otherwise omit
\end{abtandapage}

\prefacesection{Abstract}
   \input{abstract}
\end{document}
\end{verbatim}

\newpage
\noindent{\bf Example Abstract with Title and Approval Page}
\addcontentsline{toc}{subsection}{Example Abstract with Title and Approval Page}

\lbar

\begin{verbatim}
\documentclass {udthesis}
\begin{document}
\title{A Preliminary Evaluation of the\linebreak
       Delaware Cloud Seeding Project\linebreak
       Using a Principal Components-Based\linebreak
       Interpolation Model}
\author{Roh T. Uana}
\type{dissertation}
\degree{Doctor of Philosophy}
\majorfieldtrue\majorfield{Geography}
\degreedate{Spring 1993}

\begin{abtandapage}
\abprof{Theodore Q. Mentor, Ph.D.}
\end{abtandapage}

\prefacesection{Abstract}
\input{abstract}
\end{document}
\end{verbatim}

\lbar

\newpage
\section*{Senior Thesis Title and Approval Page}
\addcontentsline{toc}{section}{Senior Thesis Title and Approval Page}

To format a senior thesis you need to use
\verb+documentclass [seniorthesis] {udthesis}+ command.  


Some changes have been made to the macros to produce a correct senior
title and approval page. Generally, there is some standard text
associated with each command which appears only if you include the
command.  Note that all the commands are in lower case; this is how
they must appear in your document or an error will occur.  The commands
are

\lbar
\bskip
\noindent {\bf Title Page}:
\begin{verbatim}

\title{Title of paper}
         % To indicate a new line in the title, use \linebreak
         % at the end of each line but the last. The title is converted to 
         % upper case automatically.    
\author{name of author}
\degree{name of degree} % e.g., Bachelor of Science, Bachelor of Arts
                        %       Bachelor of Civil Engineering, etc.
\majorfieldtrue or \majorfieldfalse
         % To indicate whether or not a major is to be included. 
         % To be used when degree is Bachelor of Science or 
         % Bachelor of Arts.
         % Default is \majorfieldfalse.
\majorfield{name of majorfield} % e.g., Physics
         % Include only if you used \majorfieldtrue.
\distinctiontrue or \distinctiontrue
         % To indicate whether or not a degree with distinction
         % Default is \distinctionfalse
\degreedate{date degree is to be granted} % e.g., Fall 1993
                                          %       Spring 1993
                                          %       Summer 1993
\end{verbatim}

\noindent {\bf Approval Page}:

\begin{verbatim}
\prof{name, highest degree of your thesis adviser}
\prof{name, highest degree of your second thesis adviser}
\member{name, highest degree of committee member}{department}
\ucosafhmember{name, highest degree of honors committee member}
\chair{name, highest degree of honors chair}
\end{verbatim}

\lbar

\newpage
\subsection*{Examples for Senior Thesis Title and Approval Page}
\addcontentsline{toc}{subsection}{Examples for Senior Thesis Title and Approval Page}
\addcontentsline{toc}{subsubsection}{Degree with Distinction}

\noindent {\bf Degree with Distinction}

\begin{verbatim} 

\title{Low Temperature Specific Heat of\\
	   Amorphous Nickel-Boron Alloys}
\author{Robert Motsay}
\degree{Bachelor of Science}
\majorfieldtrue\majorfield{Physics} % optional
\distinctiontrue
\degreedate{May 1993}

\maketitlepage 

\begin{approvalpage} 
\prof{David G. Onn, Ph.D.}     
\member{Ferd Williams, Ph.D.}{Department of Physics and Astronomy}
\ucosafhmember{Arnold Kerr, Ph.D.}
\chair{Michael P. Rewa, Ph.D.}
\end{approvalpage} 

\begin{front} 

\prefacesection{Acknowledgments}
   \input{acknowl}

\prefacesection{Preface}
   \input{dedicat}

\maketocloflot

\prefacesectiontoc{Abstract}
   \input{abstract}

\end{front}
\end{verbatim}

\addcontentsline{toc}{subsubsection}{Honors Degree}

\newpage
\noindent{\bf Honors Degree}

\begin{verbatim} 

\title{Literary Theory in the Critical\\
	   Writings of Virginia Woolf}
\author{Jane J. Doe}
\degree{Bachelor of Arts}
\majorfieldtrue\majorfield{English} % optional
\degreedate{May 1993}

\maketitlepage 

\begin{approvalpage} 
\prof{Bonnie K. Scott, Ph.D.}     
\member{Hans-Peter Breuer, Ph.D.}{Department of English}
\ucosafhmember{Patricia Leighten, Ph.D.}
\chair{Robert F. Brown, Ph.D.}
\end{approvalpage} 

\begin{front}

\prefacesection{Acknowledgments}
   \input{acknowl}

\prefacesection{Preface}
   \input{dedicat}

\maketocloflot

\prefacesectiontoc{Abstract}
   \input{abstract}

\end{front}
\end{verbatim}

\chapter*{Appendix\\
           More Example Files}
\addcontentsline{toc}{chapter}{Appendix:  More Example Files}
\label{exfiles}

\lbar
\noindent {\bf main.tex}

\begin{verbatim}

\documentclass {udthesis}
\begin{document}
\include{tap}
\include{chap1}
\include{chap2}
\include{chap3}
\include{ref}
\end{document}
\end{verbatim}

This example shows a main file that includes a bibliography you create 
yourself using the {\bf thebibliography} environment.

\lbar
\noindent {\bf main.tex}

\begin{verbatim}

\documentclass {udthesis}
\begin{document}
\include{tap}
\include{chap1}
\include{chap2}
\include{chap3}
\bibliography{com}
\bibliographystyle{plain}
\end{document}
\end{verbatim}

This example shows a main file that includes a bibliography created using
\BIBTeX.
\label{main2}

\lbar

\pagebreak

\lbar
\noindent {\bf chap1.tex}

\begin{verbatim}

\chapter{Introduction}
\section{The Delaware Cloud Seeding Project}
\subsection{Historical Background}
Attempts at weather modification from
aircraft, through the introduction of
foreign material into clouds, are
relatively recent phenomena.  It was a
dark and stormy night ...

Since clouds were first seeded, at
least .......
\end{verbatim}
\lbar

\noindent {\bf ref.tex}

\begin{verbatim}

\renewcommand{\bibname}{References}
\begin{thebibliography}{999}
\bibitem {Bra1} Brahm, R.R., Jr.  1979.  ``Field Experimentation in 
                Weather Modification.'' {\it Journal of the American 
                Statistical Association}, 74(365): 57-68.

\bibitem {Catt1} Cattell, R.B.  1966.  ``The Screen Test for the Number 
                 of Factors.'' {\it Multivariate Behavioral Research}, 
                 1: 245-276.
\end{thebibliography}
\end{verbatim}

\lbar

The Title of {\bf thebibliography} will be {\bf REFERENCES}

\lbar

\noindent {\bf abstract.tex}

\begin{verbatim}

An empirical method, based on principal
components analysis for the spatial 
interpolation of daily rainfall,
is described and tested using...
\end{verbatim}
\lbar

\end{document}
